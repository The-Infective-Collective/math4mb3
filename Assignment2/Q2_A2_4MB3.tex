\documentclass{article}
\usepackage{amsmath}
\usepackage[utf8]{inputenc}
\newcommand{\R}{{\mathcal R}}
\begin{document}
Math 4MB3 - Assignment 2 Question 2
\section*{Part a}
Let mortality be denoted by $M(t)$ and prevalence be denoted by $I(t)$. If we assume that both mortality and prevalence grow exponentially, then we can write
\begin{equation*}
\begin{aligned}
M(t) = a e^{bt}
\end{aligned}
\end{equation*}

and
\begin{equation*}
\begin{aligned}
I(t) = g e^{ht}
\end{aligned}
\end{equation*}
for some constants $a$, $b$, $g$, and $h$. Furthermore, if we assume that $I(t) = \eta M(t - \tau)$, for some $\eta$ and $\tau$, then we can write
\begin{equation*}
\begin{aligned}
g e^{ht} = I(t) = \eta M(t - \tau) = \eta a e^{bt}.
\end{aligned}
\end{equation*}
Then $g e^{ht} = \eta a e ^{bt}$, and
\begin{equation*}
\begin{aligned}
e^{(h-b)t} = \frac{\eta a }{g}.
\end{aligned}
\end{equation*}
Notice that the RHS of this equation is constant and thus forces $h - b = 0$. That is, $h=b$, and thus both $I(t)$ and $M(t)$ have the same exponential growth rate.

\section*{Part b}
To determine coefficients, we restricted the data to the portion in which the semi-log plot looks approximately linear, and fit a linear model using the \texttt{lm()} function to the log-transformed data. In order to generate a more intuitive plot, this was also done for data transformed in log base 10. The corresponding intercepts and slopes for these two fits are given in the table below.
\begin{center}
\begin{tabular}{c c c}\\
			& slope & intercept \\
 \hline
$\log_e$ 	& 		& \\
$\log_{10}$ &		&
\end{tabular}
\end{center}
%---------------------------------------------
%----------------------------------------------
\section*{Part c}
We recall that the SIR model is given by Equation \ref{eq:SIRmodel}.
\begin{equation}
\label{eq:SIRmodel}
\begin{aligned}
\frac{dS}{dt}&=-\beta SI \\
\frac{dI}{dt}&=\beta SI - \gamma I \\
\frac{dR}{dt}&=\gamma I 
\end{aligned}
\end{equation}
Given that we are examining data where $I$ is small, we can use the assumption that $S\simeq 1$. Thus, this yields the equation $\frac{dI}{dt}\simeq\beta I - \gamma I$. Solving this, we have Equation \ref{eq:Isolve}
\begin{equation}
\label{eq:Isolve}
I=ce^{(\beta - \gamma)t}
\end{equation}
Following from the answer in Question 2 Part a, we know that the slope of 4.31 of our fitted mortality curve in Part b is equal to the $\beta - \gamma$ slope we would have if we took the log of Equation \ref{eq:Isolve}.
\\\\
We recall from the SIR model that the constant $\R_0$ is given by the product of the transmission rate and the mean infection period. Equivalently, we have that $\R_0 = \frac{\beta}{\gamma}$. Thus, in order to establish a value for $\R_0$, we need either $\gamma$ or $\beta$, as then we can solve for the other variable using that $\beta - \gamma = 4.31$. It is much more logical to look for an independent measure of $\gamma$, as the mean infectious period given by $\frac{1}{\gamma}$ is easier to infer from data than the transmission rate. 
\\\\
If in our case, the mean infectious period ($\frac{1}{\gamma}$) is 4 days, then we know that $\gamma = \frac{1}{4}$. Thus, we have that $\beta=4.56$. From this information, we can solve for $\R_0 = 4.56*4 = 18.24$.
\end{document}
