\documentclass[12pt]{article}
\usepackage{scrtime} % for \thistime (this package MUST be listed first!)
\usepackage[margin=1in]{geometry}
\usepackage{listings}


%% Language and font encodings
\usepackage[english]{babel}
\usepackage[utf8x]{inputenc}
\usepackage[T1]{fontenc}

%% Useful packages
\usepackage{amsmath}
\usepackage{amsthm} % for theorems and proofs
\usepackage{amsfonts} % mathbb
\usepackage{graphicx}
\usepackage[colorinlistoftodos]{todonotes}
\definecolor{aqua}{RGB}{0, 128, 225}
\usepackage[colorlinks=true,citecolor=aqua,linkcolor=aqua,urlcolor=aqua]{hyperref}
\usepackage[nameinlink]{cleveref}

\usepackage{fancyhdr,lastpage}
\pagestyle{fancy}
\fancyhf{} % clear all header and footer parameters
%%%\lhead{Student Name: \theblank{4cm}}
%%%\chead{}
%%%\rhead{Student Number: \theblank{3cm}}
%%%\lfoot{\small\bfseries\ifnum\thepage<\pageref{LastPage}{CONTINUED\\on next page}\else{LAST PAGE}\fi}
\lfoot{}
\cfoot{{\small\bfseries Page \thepage\ of \pageref{LastPage}}}
\rfoot{}
\renewcommand\headrulewidth{0pt} % Removes funny header line

\newcommand{\R}{{\cal R}}

\title{Math 4MB3: Draft Group Project - Supplementary file}
\author{\underline{\emph{Group Name}}: \texttt{{\color{blue}The Infective Collective}}\\
{}\\
\underline{\emph{Group Members}}: {\color{blue}Aurora Basinski-Ferris, Michael Chong, Daniel Park, Daniel Presta}}

\date{\today\ @ \thistime}


\begin{document}
\maketitle

\section{Construction of the model}

First, consider the following coupled seasonally forced SIR model consisting of $n$ patches:
\begin{equation}
\begin{aligned}
\frac{dS_i}{dt} &= \mu N - \left(\sum_{j=1}^n \beta(t) m_{ij} I_j + \mu\right) S_i \\
\frac{dI_i}{dt} &= \left(\sum_{j=1}^n \beta(t) m_{ij} I_j + \mu\right) S_i - (\mu  + \gamma) I_i\\
\frac{dR_i}{dt} &= \gamma I_i - \mu R_i
\end{aligned}
\end{equation}
where $S_i$, $I_i$ and $R_i$ are numbers of susceptible, infected and recovered individuals in patch $i$, respectively. 
$N$ is population size in every patch and is assumed to be constant. 
$\beta(t)$ is the transmission rate, $\gamma$ is the per capita recovery rate and $\mu$ is the per capita death rate as well as birth rate. 
$m_{ij}$ is the proportion of contacts from patch $j$ that are dispersed to patch $i$. 
The dispersal matrix is then given by $M = \left[m_{ij}\right]$ and we have $\sum_{i=1}^n m_{ij} = 1$. 
To allow for seasonal forcing, we model the transmission rate using a trigonometric function [CITE]:
\begin{equation}
\beta(t) = b_0 (1 + b_1 \cos(2 \pi t))
\end{equation}

Note that a susceptible individual in patch $i$ leaves the susceptible compartment at rate $\sum_{j=1}^n \beta(t) m_{ij} I_j + \mu$. We can view this quantity as hazard and obtain the probability that a susceptible individual survives during a time interval $(t, t + \Delta t)$:
\begin{equation}
\exp \left(-\int_{t}^{t + \Delta t} \sum_{j=1}^n \beta(s) m_{ij} I_j(s) + \mu d s \right),
\end{equation}
where for sufficiently small $\Delta t$, we can write
\begin{equation}
\int_{t}^{t + \Delta t} \sum_{j=1}^n \beta(s) m_{ij} I_j(s) + \mu d s \approx \left(\sum_{j=1}^n \beta(t) m_{ij} I_j(s) + \mu \right) \Delta t.
\end{equation}
Rearranging, number of individuals that leave the susceptible compartment after $\Delta$ time step is given by
$$
S_{i, \tiny{\textrm{leave}}}(t) = \left(1 - \exp \left(-\left(\sum_{j=1}^n \beta(t) m_{ij} I_j(s) + \mu \right) \Delta t \right) \right) S_i(t)
$$
Assuming that the transmission rate and number of infected individuals stays constant over the interval $(t, t+\Delta)$, the probability that a susceptible individual leaves the compartment due to infection is given by 
\begin{equation}
\frac{\sum_{j=1}^n \beta(t) m_{ij} I_j(s)}{\sum_{j=1}^n \beta(t) m_{ij} I_j(s) + \mu}
\end{equation}
Incidence (i.e., number of suseptible individuals that become infected during $(t, t+\Delta t)$ interval) is then given by the product of above probability and $S_{i, \tiny{\textrm{leave}}}(t)$.
Similarly, an infected individual and a recovered individual suffer from constant hazard of $\gamma + \mu$ and $\mu$, respectively, and these can each be translated into survival probability, yielding
\begin{equation}
\begin{aligned}
I_{i, \tiny{\textrm{leave}}}(t) &= (1 - \exp(-(\gamma + \mu)\Delta t)) I_i(t) \\
R_{i, \tiny{\textrm{leave}}}(t) &= (1 - \exp(-\mu \Delta t)) R_i(t)
\end{aligned}
\end{equation}
Then, the full discrete time model is given by
\begin{equation}
\label{eq:discretedeterministic}
\begin{aligned}
S_{k}(t+\Delta t) &= b_k(t) + S_k(t) - S_{k, \tiny{\textrm{leave}}}(t)\\
I_{k}(t+\Delta t) &= i_k(t) + I_k(t) - I_{k, \tiny{\textrm{leave}}}(t)\\
R_{k}(t+\Delta t) &= r_k(t) + R_k(t) - R_{k, \tiny{\textrm{leave}}}(t)
\end{aligned}
\end{equation}
where $b_k(t)$, $i_k(t)$, and $r_k(t)$ represent number of new susceptible, infected, and recovered individuals that are produced between the interval $(t, t+\Delta t)$:
\begin{equation}
\begin{aligned}
i_k(t) &= \frac{\sum_{j=1}^n \beta(t) m_{ij} I_j(s)}{\sum_{j=1}^n \beta(t) m_{ij} I_j(s) + \mu} S_{k, \tiny{\textrm{leave}}}(t)\\
r_k(t) &= \frac{\gamma}{\gamma + \mu} R_{k, \tiny{\textrm{leave}}}(t)\\
b_k(t) &= S_{k, \tiny{\textrm{leave}}}(t) - i_k(t) + I_{k, \tiny{\textrm{leave}}}(t) - r_k(t)\\
\end{aligned}
\end{equation}

\bibliographystyle{vancouver}
\bibliography{synchrony}
\end{document}