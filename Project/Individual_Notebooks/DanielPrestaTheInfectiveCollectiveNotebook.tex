\documentclass[12pt]{article}\usepackage[]{graphicx}\usepackage[]{color}
%% maxwidth is the original width if it is less than linewidth
%% otherwise use linewidth (to make sure the graphics do not exceed the margin)
\makeatletter
\def\maxwidth{ %
  \ifdim\Gin@nat@width>\linewidth
    \linewidth
  \else
    \Gin@nat@width
  \fi
}
\makeatother

\definecolor{fgcolor}{rgb}{0.345, 0.345, 0.345}
\newcommand{\hlnum}[1]{\textcolor[rgb]{0.686,0.059,0.569}{#1}}%
\newcommand{\hlstr}[1]{\textcolor[rgb]{0.192,0.494,0.8}{#1}}%
\newcommand{\hlcom}[1]{\textcolor[rgb]{0.678,0.584,0.686}{\textit{#1}}}%
\newcommand{\hlopt}[1]{\textcolor[rgb]{0,0,0}{#1}}%
\newcommand{\hlstd}[1]{\textcolor[rgb]{0.345,0.345,0.345}{#1}}%
\newcommand{\hlkwa}[1]{\textcolor[rgb]{0.161,0.373,0.58}{\textbf{#1}}}%
\newcommand{\hlkwb}[1]{\textcolor[rgb]{0.69,0.353,0.396}{#1}}%
\newcommand{\hlkwc}[1]{\textcolor[rgb]{0.333,0.667,0.333}{#1}}%
\newcommand{\hlkwd}[1]{\textcolor[rgb]{0.737,0.353,0.396}{\textbf{#1}}}%
\let\hlipl\hlkwb

\usepackage{framed}
\makeatletter
\newenvironment{kframe}{%
 \def\at@end@of@kframe{}%
 \ifinner\ifhmode%
  \def\at@end@of@kframe{\end{minipage}}%
  \begin{minipage}{\columnwidth}%
 \fi\fi%
 \def\FrameCommand##1{\hskip\@totalleftmargin \hskip-\fboxsep
 \colorbox{shadecolor}{##1}\hskip-\fboxsep
     % There is no \\@totalrightmargin, so:
     \hskip-\linewidth \hskip-\@totalleftmargin \hskip\columnwidth}%
 \MakeFramed {\advance\hsize-\width
   \@totalleftmargin\z@ \linewidth\hsize
   \@setminipage}}%
 {\par\unskip\endMakeFramed%
 \at@end@of@kframe}
\makeatother

\definecolor{shadecolor}{rgb}{.97, .97, .97}
\definecolor{messagecolor}{rgb}{0, 0, 0}
\definecolor{warningcolor}{rgb}{1, 0, 1}
\definecolor{errorcolor}{rgb}{1, 0, 0}
\newenvironment{knitrout}{}{} % an empty environment to be redefined in TeX

\usepackage{alltt}
\usepackage{scrtime} % for \thistime (this package MUST be listed first!)
\usepackage[margin=1in]{geometry}
\usepackage{fancyhdr}
\pagestyle{fancyplain}
\usepackage[T1]{fontenc}
\usepackage{tikz-cd}
\usepackage{amsmath}
% So David can comment:
\newcommand{\de}[1]{{\color{red}{\bfseries DE:} #1}}
% Use the R logo when referring to R:
\usepackage{xspace}
\newcommand{\Rlogo}{\protect\includegraphics[height=2ex,keepaspectratio]
{images/Rlogo.pdf}\xspace} 
% Note: must have the R logo saved in
%       an "images" directory



\rhead{\fancyplain{}{4MB3 Project Notebook 2018 \hfill Daniel Presta}}
\title{Math 4MB3 Project Notebook 2018}
\author{Daniel Presta (The Infective Collective)}
\date{\today\ @ \thistime}
\IfFileExists{upquote.sty}{\usepackage{upquote}}{}
\begin{document}
\maketitle

\section*{Friday 9 March 2018}

\textbf{Group Meeting (in class)} \\
\emph{Approximate Duration: 0.5 Hours}

\begin{itemize}
\item We discussed project topics, and settled on ``Spatial Epidemic Dynamics: Synchronization" as our topic
\item We sent an email to Dr. Earn, confirming our topic choice.
\end{itemize}

\section*{Wednesday 14 March 2018}

\textbf{Group Meeting (in class)} \\
\emph{Approximate Duration: 1  Hour}

\begin{itemize}
\item Updated ``README.md" in the GitHub repository; added papers that should be read for the next meeting.
\item Discuss step (b) (construction of spatial SIR model) next meeting, will hopefully have model approved by Dr. Earn by the end of the next meeting.
\item Read suggested papers.
\end{itemize}

\section*{Friday 16 March 2018}

\textbf{Group Meeting (in class)} \\
\emph{Approximate Duration: 1 Hour}

\begin{itemize}
\item Tried to construct spatial SIR model after reading suggested papers.
\item We approached Dr. Earn about a possible idea for a model.
\end{itemize}

\textbf{Group Meeting} \\
\emph{Approximate Duration: 1 Hour}

\begin{itemize}
\item Split up work 
\item Aurora and Michael are to work on numerical/computational simulations.
\item DPark and DPresta will try to connect this model to existing theorems
\item DPark and DPresta will try to use existing theorems to analytically determine/prove a set of criteria for coherence in our model.
\end{itemize}

\section*{Tuesday 20 March 2018}

\textbf{Solo Work} \\
\emph{Approximate Duration: 2 Hours}

\begin{itemize}
\item Read (Earn \& Levin 2006) and (McCluskey \& Earn 2011) to better understand analytical approaches
\item Using their approaches, attempted to derive a set of criteria for coherence in our model.
\item Messed around with possible reproductive functions, before stopping and realizing that the dispersal matrix in the exponent cannot be simplified.
\end{itemize}

\section*{Wednesday 21 March 2018}

\textbf{Solo Work} \\
\emph{Approximate Duration: 2 Hours}

\begin{itemize}
\item Took different approach to finding analytical condition.
\item Like DPark, I analyzed a two patch SIS model and tried to derive a coherence criterion.
\item Difference is that I did not assume equal coupling.
\item Algebra did not simplify, ran into a wall.
\end{itemize}

\textbf{Group meeting} \\
\emph{Approximate Duration: 1 Hour}
\begin{itemize}
\item Aurora, Michael, and DPark all troubleshooted code and fixed their error in the Rcpp file.
\item I worked on simplifying the algebra required to derive a coherence criterion for a two patch SIS system (where dispersal matrix does not have equal coupling).
\end{itemize}

\section*{Thursday 22 March 2018}

\textbf{Solo work} \\
\emph{Approximate Duration: 1.5 Hours}
\begin{itemize}
\item Attempted to find coherence criterion for continuous time version of our SIR model (first examined the continuous time version of the SIS model).
\item Once again assumed that dispersal matrix did not have equal coupling.
\item Algebra became impossible to simplify; looking like analysis may not be as easy we thought. Only results obtained so far come from basic two patch SIS models.
\end{itemize}

\section*{Friday 23 March 2018}

\textbf{Group Meeting (in class)} \\
\emph{Approximate Duration: 1 Hour}

\begin{itemize}
\item Split up work for draft.
\item Decided that DPark and I will be doing numerical work now, while Aurora and Michael will be doing analytical work.
\end{itemize}

\section*{Sunday 25 March 2018}

\textbf{Solo work} \\
\emph{Approximate Duration: 1.5 Hours}
\begin{itemize}
\item Learned basics of Rcpp so that I can write a source file in Rcpp for a stochastic SIR model.
\item Began to tinker with Rcpp source file for our simple SIR model (no alterations); will write code tomorrow
\end{itemize}

\section*{Monday 26 March 2018}

\textbf{Group Meeting (in class)} \\
\emph{Approximate Duration: 1 Hour}
\begin{itemize}
\item Discussed direction for the paper and further split up work.
\item Analytical work will be shelved, for now; only focusing on numerical simulations.
\item I will be working on the investigation of effects of seasonal forcing and the changes in seasonal amplitude.
\end{itemize}

\textbf{Solo Work} \\
\emph{Approximate Duration: 2.5 Hours}
\begin{itemize}
\item Wrote a function in Rcpp that simulates changes for a stochastic version of our SIR model.
\item This file was then used by DPark and he copied the important bits into the "SIRmodelnpatch.cpp" file, in order to keep repository clean and code easy to work with.
\item Began to analyze different functions for $\beta (t)$. Will use sinusoidally-forced transmission rate, will also use a term-time transmission rate similar to one seen in (He et al. 2009).
\end{itemize}


\textbf{Group Meeting (evening)} \\
\emph{Approximate Duration: 1 Hour}
\begin{itemize}
\item Discussed plan for the draft.
\end{itemize}

\section*{Tuesday 27 March 2018}

\textbf{Solo work} \\
\emph{Apprxomiate Duration: 3.5 Hours}
\begin{itemize}
\item Wrote a function for term-time forced transmission rate and added it to the general source file ("SIRmodelnpatch.cpp").
\item Encountered several errors; prevalence vs. time graph looked extremely incorrect.
\item Also simulated various runs of sinusoidally-forced transmission rate when seasonal amplitude is changed, observed trends.
\item Wrote a paragraph in introduction.
\end{itemize}

\textbf{Group meeting} \\
\emph{Approximate Duration: 1 Hour}
\begin{itemize}
\item DPark and Michael helped fix the bug in my code, turned out it was a simple int/float conflict error in Rcpp.
\item Agreed that everyone will have things done by 9 AM tomorrow.
\end{itemize}

\textbf{Solo work} \\
\emph{Apprxomiate Duration: 3 Hours}
\begin{itemize}
\item Adapted the code previously used by Aurora and Michael to account for changes in seasonal amplitude (and ignore changes in $\mathcal{R}_0$ and connectivity matrices).
\item Ran simulations for different seasonal amplitude values for both forced transmission rates. Coherence trends are difficult to detect right now. A bifurcation diagram will be necessary to better understand the effects of seasonal amplitude.
\item Wrote up my findings in the project document, while also explaining a bit about time-dependent transmission rates in general.
\end{itemize}

\section*{Sunday 1 April 2018}

\textbf{Solo work} \\
\emph{Apprxomiate Duration: 1 Hour}
\begin{itemize}
\item Read papers on coherence, specifically Earn et. al (2000) to determine next steps that should be taken for stochastic results.
\item Tried to make a diagram similar to that displayed in stohasticfig for only two mixing rates (but for more values of basic reproductive number), but stopped once I realized that the simulation would take an extremely long time.
\end{itemize}

\section*{Monday 2 April 2018}

\textbf{Group work} \\
\emph{Apprxomiate Duration: 1 Hour}
\begin{itemize}
\item We met to discuss Dr. Earn's feedback and to settle on a definitive path for the remainder of the project.
\item Dr. Earn gave us feedback on how we should approach the remainder of the paper.
\item Split up work for the next couple days; I will be running sample trajectories and exploring the behaviour of stochastic trajectories near cycle endpoints of the bifurcation diagram. 
\end{itemize}

\textbf{Solo work} \\
\emph{Apprxomiate Duration: 1 Hour}
\begin{itemize}
\item Played around with sample trajectories for a bit (for the stochastic model).
\item Specifically tried to examine the correspondence between sharp increases and decreases in local/global extinction and the beginning of new cycles (or end of old cycles) in the bifurcation diagram.
\end{itemize}

\section*{Tuesday 3 April 2018}

\textbf{Solo work} \\
\emph{Apprxomiate Duration: 3 Hours}
\begin{itemize}
\item Continued to examine various sample trajectories; continued work from previous night.
\item Using a few of these samples, I made figures that displayed local/global extinction and rescue effect, as well as synchronous vs. asynchronous behaviour.
\item Still need to learn \texttt{ggplot} so I can make my graphs look consistent with other figures in the project.
\end{itemize}

\textbf{Group work} \\
\emph{Apprxomiate Duration: 1 Hour}
\begin{itemize}
\item We met to talk about our progress on our individual parts and to plan the presentation.
\item Set up a \texttt{beamer} file and formed a general idea of what should be on each slide.
\item Split up work for presentation and decided which plots we must have done for Friday (presentation day).
\end{itemize}

\section*{Wednesday 4 April 2018}

\textbf{Group work} \\
\emph{Apprxomiate Duration: 1 Hour}
\begin{itemize}
\item Changed our incoherence measure; it is no longer an absolute threshold, it is instead relative to epidemic size.
\item Discussed themes for our plots and how to go about making our indvidual slides look consistent.
\end{itemize}

\textbf{Group work} \\
\emph{Apprxomiate Duration: 1 Hour}
\begin{itemize}
\item Changed our incoherence measure again; changed it back to an absolute threshold, but instead raised the threshold.
\item Finished planning out presentation, I will work on the introductory slides with Aurora and stochastic results slides with DPark.
\end{itemize}

\textbf{Solo work} \\
\emph{Apprxomiate Duration: 2 Hours}
\begin{itemize}
\item Learned basics from \texttt{ggplot} from DPark so I could make my graphs look nicer.
\item DPark and I worked on slideshow presentation, we made our graphs look consistent and decided which figures to include in presentation, as well as what to write.
\item Began introductory slide of presentation.
\end{itemize}

\section*{Thursday 5 April 2018}

\textbf{Solo work} \\
\emph{Apprxomiate Duration: 1 Hour}
\begin{itemize}
\item Made some edits to stochastic section of presentation.
\item Finished introductory slides with Aurora.
\end{itemize}

\textbf{Group work} \\
\emph{Apprxomiate Duration: 3.5 hours}
\begin{itemize}
\item Made final changes to the \texttt{beamer} file and finalized the presentation for tomorrow.
\item Discussed changing coherence measure again but decided against it.
\item Changed research questions and introduction a bit, also talked about how to transition from section to section to maintain good flow during the presentation.
\item Ran through the presentation a few time; each time we practiced, we gave each other feedback on what we could do to shorten our speaking time and have better flow.
\end{itemize}

\textbf{Solo work} \\
\emph{Apprxomiate Duration: 0.5 Hours}
\begin{itemize}
\item Practiced my parts of presentation, I will be doing the introduction, summary, and conlusion/future questions.
\end{itemize}

\section*{Friday 6 April 2018}

\textbf{Group work} \\
\emph{Apprxomiate Duration: 1 hour}
\begin{itemize}
\item Practiced the presentation a few times before presenting in class.
\item Discussed how to present next questions/future directions, and how to try our best to stay under time.
\end{itemize}

\textbf{Solo work} \\
\emph{Apprxomiate Duration: 0.5 Hours}
\begin{itemize}
\item Practiced my parts of presentation in between our group meeting and our presentation in class.
\end{itemize}

\section*{Saturday 7 April 2018}

\textbf{Solo work} \\
\emph{Apprxomiate Duration: 3.5 Hours}
\begin{itemize}
\item Wrote up the first bit (paragraph and Figure 2) of the stochastic results section.
\item Later, I added more to the stochastic results section and edited the rest of the section.
\item I fixed Figure 2 so that its axis titles would be the same size as those in other figures.
\item I began to read other papers to get an idea of what to write for the introduction.
\end{itemize}

\section*{Sunday 8 April 2018}

\textbf{Solo work} \\
\emph{Apprxomiate Duration: 7.5 Hours}
\begin{itemize}
\item Finished up edits for stochastic section and discussion.
\item Wrote introduction and abstract.
\item Contributed to the supplementary file by writing the section of time-dependent transmission rate and the section on observed synchrony and asynchrony in stochastic models.
\item Read through final document a couple times to make a few small edits, also helped organize word count.
\end{itemize}

\textbf{Group work} \\
\emph{Apprxomiate Duration: 5 hours}
\begin{itemize}
\item Met over Google Hangout chat twice; first meeting was for two hours, second meeting was for three hours.
\item During the first meeting, we went through our document together, and made edits together as a group.
\item We split up any edits and final work and decided to meet again the evening.
\item In the evening, we edited the entire document again, specifically going over parts that were added since the afternoon.
\item By the end of the second meeting, we were more or less ready to submit.
\end{itemize}

\section*{Total time spent on this project}

\begin{quote}

\emph{Group work:} 22 hours

\emph{Solo work:} 36 hours

\end{quote}

\end{document}
